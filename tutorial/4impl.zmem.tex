\subsection{Acesso à memória SRAM}
\label{sec:impl.mem}

Nosso módulo utiliza a memória SRAM para armazenar as portas bloqueadas
e decidir quais pacotes filtrar.  No sexto estado do processamento de um
pacote emitimos uma requisição de leitura para o endereço
\ssf{SRAM\_PORTS\_ADDR}, que contem as portas TCP bloqueadas.  No sétimo
estado esperamos a leitura completar e então utilizamos o dado lido para
verificar se o pacote precisa ser descartado ou não.

Nosso \emph{firewall} emite operações de leitura da memória para o
módulo \ssf{sram\_arbiter}, que intermedia o acesso à memória SRAM.  As
linhas de comunicação do nosso \emph{firewall} com o \ssf{sram\_arbiter}
ilustram a interface de acesso à memória.

\begin{verilogcode}
      output reg                       sram_rd_req,
      output reg [SRAM_ADDR_WIDTH-1:0] sram_rd_addr,
      input [DATA_WIDTH-1:0]           sram_rd_data,
      input                            sram_rd_ack,
      input                            sram_rd_vld,
      output reg                       sram_wr_req,
      output reg [SRAM_ADDR_WIDTH-1:0] sram_wr_addr,
      output reg [DATA_WIDTH-1:0]      sram_wr_data,
      input                            sram_wr_ack,
\end{verilogcode}

O \ssf{sram\_arbiter} pode receber uma requisição de leitura ou escrita
por ciclo de relógio.  Requisições de leitura são indicadas ligando
\ssf{sram\_rd\_req} e informando o endereço a ser lido em
\ssf{sram\_rd\_addr}.  No próximo ciclo de relógio o \ssf{sram\_arbiter}
indica se a requisição foi recebida com sucesso ligando o sinal
\ssf{sram\_rd\_ack}.  Como leituras demoram alguns ciclos para serem
atendidas, o módulo que pediu a leitura deve esperar os dados serem
retornados e disponibilizados pelo \ssf{sram\_arbiter}.  O
\ssf{sram\_arbiter} informa que os dados estão disponíveis em
\ssf{sram\_rd\_data} ligando \ssf{sram\_rd\_vld}.

Requisições de escrita são indicadas ligando \ssf{sram\_wr\_req},
informando o endereço a ser escrito em \ssf{sram\_wr\_addr} e informando
o dado a ser escrito em \ssf{sram\_wr\_data}.  O \ssf{sram\_arbiter}
indica se a requisição foi recebida com sucesso ligando o sinal
\ssf{sram\_wr\_ack}.  Como o dado a ser escrito é armazenado pelo
\ssf{sram\_arbiter}, o módulo que fez a requisição de escrita não
precisa esperar mais nenhuma confirmação do \ssf{sram\_arbiter}.  Se
ambos os sinais \ssf{sram\_rd\_req} e \ssf{sram\_wr\_req} estiverem
ligados, nosso \ssf{sram\_arbiter} prioriza a requisição de escrita.  A
SRAM usada na NetFPGA garante que leituras realizadas após escritas
lerão o dado atualizado.

A interface que exportamos em nosso \ssf{sram\_arbiter} é simplificada.
Como descrito na seção~\ref{sec:arch.hw}, a NetFPGA possui dois bancos
de memórias SRAM, cada um com $2^{19}$ linhas de 36~bits.  Nosso
\ssf{sram\_arbiter} combina os dois bancos para apresentar uma abstração
de memória de $2^{19}$ linhas de 64~bits.  Usamos 8~bits de cada linha
como bits de paridade, calculados e verificados automaticamente pelo
\ssf{sram\_arbiter}.

\begin{verilogcode}
   // sram_arbiter.v
   generate
      genvar m;
      for(m = 0; m < 8; m = m+1) begin: calc_par_bits
      assign parbit[m] = wr_data[m*8] ^ wr_data[m*8+1] ^
            wr_data[m*8+2] ^ wr_data[m*8+3] ^ wr_data[m*8+4] ^
            wr_data[m*8+5] ^ wr_data[m*8+6] ^ wr_data[m*8+7];
      end // wr_data is 64 bits wide
   endgenerate 
   generate
      genvar l;
      for(l = 0; l < 8; l = l+1) begin: expand_wr_data
         assign wr_data_exp[(l+1)*9-1 : l*9] =
            {wr_data[(l+1)*8-1:l*8], parbit[l]};
         end // wr_data_exp is 72 bits wide (36*2)
   endgenerate
\end{verilogcode}

Para acessar as duas memórias simultaneamente duplicamos os sinais de
requisição de escrita ou leitura e os endereços para os dois bancos de
memória.  Para requisições de escrita escrevemos metade dos dados em
cada banco e para requisições de leitura concatenamos os dados dos dois
bancos.  Abaixo mostramos o código para realizar estas operações.  Este
código fica dentro do módulo \ssf{nf2\_core}.  O módulo \ssf{nf2\_core}
é o módulo raiz do \emph{software} da NetFPGA e comunica diretamente com
os pinos do FPGA.  O \ssf{nf2\_core} conecta os pinos do FPGA conectados
às memórias SRAM ao \ssf{sram\_arbiter} da seguinte forma.

\begin{verilogcode}
// nf2_core.v
// hardware pins       sram_arbiter
assign sram1_wr_data = wr_data_exp[`SRAM_DATA_WIDTH-1:0];
assign sram2_wr_data = wr_data_exp[2*`SRAM_DATA_WIDTH-1:`SRAM_DATA_WIDTH];
assign sram1_we      = sram_we; // 0 for write, 1 for read
assign sram2_we      = sram_we;
assign sram1_addr    = sram_addr;
assign sram2_addr    = sram_addr;
// sram_arbiter        hardware pins
assign sram_rd_data  = {sram2_rd_data, sram1_rd_data};
\end{verilogcode}

O \ssf{sram\_arbiter} pode ser modificado para permitir acesso mais
eficiente à memória caso a aplicação tenha um padrão específico de
acessos.  Por exemplo, é possível modificar as atribuições acima para
permitir ler endereços distintos em cada banco de SRAM.  A SRAM também
provê um mecanismo para permitir escritas parciais, escolhendo quais
bytes devem ser escritos em uma requisição de escrita.\footnotemark{}

\footnotetext{Não mostramos esta funcionalidade no texto.  Nossa
implementação não suporta escritas parciais.  Escritas parciais poderiam
ser controladas configurando o valor das linhas \sssf{sram\_bw} no
\sssf{sram\_arbiter}.}

Para exemplificar o controle de acesso à SRAM num nível mais baixo,
iremos explicar o tratamento de uma requisição de leitura (requisições
de escrita são mais simples).  Quando o \ssf{sram\_arbiter} recebe uma
requisição de leitura, ele desabilita escrita ligando o sinal
\ssf{sram\_we} (este sinal possui lógica negativa), repassa o endereço a
ser lido ao \emph{hardware} e confirma a requisição de leitura.

\begin{verilogcode}
   // sram_arbiter.v
   else if(sram_rd_req) begin
      hw_we <= 1'b1;                // read
      hw_addr <= sram_rd_addr;
      sram_rd_ack <= sram_rd_req;   // acknowledge read request
      sram_wr_ack <= 0;             // do not acknowledge write
      rd_vld_early3 <= sram_rd_req; // data back in three cycles
      ...
   end
\end{verilogcode}

Como o dado demora dois ciclos para ser retornado da SRAM após a
requisição, o \ssf{sram\_arbiter} possui um \emph{pipeline} interno para
esperar os dados serem retornados pela SRAM.  Após dois ciclos o
\ssf{sram\_arbiter} armazena o dado lido no registrador
\ssf{sram\_rd\_data} e encaminha este registrador para o \emph{firewall}
no terceiro ciclo de relógio após a requisição.

\begin{verilogcode}
   // sram_arbiter.v
   rd_vld_early2 <= rd_vld_early3; // waited 1
   rd_vld_early1 <= rd_vld_early2; // waited 2
   if(rd_vld_early1) begin // memory sending data this cycle, storing
      if(parity_check)
         sram_rd_data <= rd_data_exp_parsed; // no parity bits
      else
         sram_rd_data <= 64'hdeadfeeddeadfeed;
   end
   sram_rd_vld <= rd_vld_early1;   // data is here, set valid bit
\end{verilogcode}
