\newpage
\section{Introdução}

Comutadores (\emph{switches}) e roteadores (\emph{routers}) atuais
suportam uma quantidade limitada de arquiteturas e protocolos de
rede.  Mesmo para as arquiteturas e protocolos suportados,
comutadores e roteadores permitem processamento limitado dos
pacotes, como decremento do TTL, verificação do \emph{checksum},
balanceamento de carga e descarte.  Esta falta de flexibilidade
dificulta a avaliação e implantação de mecanismos como
OpenSketch~\cite{yu13opensketch}, de protocolos alternativos como
Free-riding Multicast~\cite{ratnasamy06multicast} e de novas
arquiteturas como XIA~\cite{han12xia}, CCN~\cite{jacobson09content},
ou SDN~\cite{casado09ethane}.  O caminho padrão para avaliação e
implantação dessas soluções seria desenvolver novas placas de rede e
processadores de pacotes específicos em \emph{hardware} (ASICs), o
que é economicamente inviável.

A NetFPGA é uma plataforma aberta que combina um FPGA
(\emph{field-programmable gate array}, ou arranjo de portas
programável em campo) memória (SRAM e DRAM) e processadores de
sinais numa placa com quatro portas Ethernet de 1\,Gbps ou 10\,Gbps.
Desenvolvedores podem programar o FPGA e utilizar a memória para
implementar novos mecanismos, protocolos e arquiteturas.  A NetFPGA
permite modificação dos pacotes em trânsito, controle total sobre o
encaminhamento e manutenção de estado.

A NetFPGA é particularmente útil como uma alternativa de
\emph{hardware} flexível e de baixo custo para avaliação de
protótipos de pesquisa.  NetFPGAs já foram utilizadas em vários
projetos de pesquisa (e.g.,~\cite{yu13opensketch,
Naous:2008:IOS:1477942.1477944, ghani10secure, thinh12fpga,
antichi12open, lombardo12netfpga}).  Comparada com a abordagem de
implementar soluções em \emph{software}, a NetFPGA garante
processamento e encaminhamento na taxa de transmissão das interfaces
(sem sacrificar banda), baixa latência e não onera a CPU do
computador.

% A plataforma NetFPGA permite processamento genérico de pacotes de
% rede em hardware.  A NetFPGA pode ser utilizada para
% desenvolvimento de protótipos de pesquisa bem como implantação de
% novas funcionalidades.

Neste texto, introduziremos o leitor à NetFPGA. Na
\secstr~\ref{sec:example} iremos demonstrar nosso \emph{firewall} em
funcionamento do ponto de vista do usuário e descreveremos como
configurar o ambiente de desenvolvimento da NetFPGA.  Na
\secstr~\ref{sec:arch} iremos apresentar as arquiteturas de
\emph{hardware} e \emph{software} da NetFPGA, dando uma visão geral
das funcionalidades e do desenvolvimento na NetFPGA.  Na
\secstr~\ref{sec:impl} iremos demonstrar passo-a-passo a criação de
um novo módulo na NetFPGA.  Usaremos como exemplo um \emph{firewall}
para filtragem de pacotes em \emph{hardware} em função da porta de
destino do protocolo TCP.  Nosso \emph{firewall} é um exemplo
didático que cobre a maior parte dos conceitos necessários para
desenvolvimento de um novo projeto, como utilização de memória bem
como processamento e modificação de pacotes.  Por fim, na
\secstr~\ref{sec:related} iremos discutir outros projetos utilizando
a NetFPGA.  Acreditamos que os conceitos do minicurso irão permitir
pesquisadores e programadores utilizarem a NetFPGA com módulos
existentes bem como desenvolverem novos módulos para processamento
de pacotes em \emph{hardware}.

