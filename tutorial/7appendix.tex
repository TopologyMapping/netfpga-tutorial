\section*{Apêndice A: Diretivas de programação}
\label{sec.impl.apendice}

Neste material adotamos algumas convenções para facilitar o
entendimento, listadas a seguir. Alguns dos exemplos citados são
recomendações de boas técnicas comuns em linguagens de programação,
enquanto outras são necessárias para que o hardware resultante
alcance o comportamento desejado do projeto.\footnote{As
recomendações foram extraídas e adaptadas de
\sssf{http://www.xilinx.com/itp/xilinx10/books/docs/sim/sim.pdf} e
\sssf{https://github.com/NetFPGA/netfpga/wiki/VerilogCodingGuidelines}.}

\begin{itemize}

\item Use constantes e parâmetros para conferir legibilidade ao
código.  Defina constantes em letra \ssf{MAIÚSCULA}.

\item Trechos do código que fazem asserções, monitoram ou imprimem
valores de variáveis para teste do projeto devem estar posicionadas
entre as diretivas \ssf{synthesis translate\_off} e \ssf{synthesis
translate\_on}. Estes trechos são executados na simulação, mas são
ignorados pelo sintetizador na criação do \emph{bitfile} do projeto.

\item Em circuitos sequenciais utilize somente atribuições
não-blocantes (\ssf{<=}) dessa forma todas as atribuições dentro do
bloco (geralmente \ssf{always}) são avaliadas primeiro e depois
realizadas na simulação e permitem ao sintetizador inferir o
circuito síncrono. No exemplo a seguir todas as atribuições são
sincronizadas com o sinal de relógio e recebem os valores anteriores
dos sinais, ou seja, a sequência não importa:

\begin{verilogcode}
always @(posedge clk) begin
    B <= A;
    C <= B;
    D <= C;
end
\end{verilogcode}

\item Em circuitos combinacionais certifique-se de preencher a lista
de sensibilidade com todas as variáveis de interesse ou usar
\ssf{always @(*)}. Nesses circuitos qualquer alteração em alguma das
variáveis na lista forçam a reexecução do bloco (em geral
\ssf{always} ou \ssf{assign}) imediatamente. Certifique-se também de
somente utilizar atribuições blocantes (\ssf{=}) para que o
sintetizador infira um circuito assíncrono. O parâmetro \ssf{@(*)} é
um recurso do Verilog que preenche automaticamente a lista de
sensibilidade com todos os valores atribuídos dentro do bloco. No
exemplo a seguir, as variáveis B, C e D recebem o valor de A:

\begin{verilogcode}
always @(*) begin
    B = A;
    C = B;
    D = C;
end
\end{verilogcode}

\item Atribua todas as variáveis em todos os estados possíveis do
seu projeto. Este erro ocorre geralmente em expressões condicionais
\ssf{if} sem a cláusula \ssf{else} ou blocos \ssf{case}. No exemplo
abaixo, \ssf{cnt\_nxt} recebe o valor atual de \ssf{cnt} por padrão,
para evitar que ele deixe de ser atribuído se algum estado não
requerer sua atualização, e \ssf{state\_nxt} é atribuído em todos os
estados. 

\item Em lógicas de controle é recomendado criar circuitos
combinatoriais que recebam os valores do próximo estado. Os valores
do próximo estado devem ser guardados em registradores em circuitos
síncronos com a borda do relógio:

\begin{verilogcode}
always @(*) begin
    cnt_nxt = cnt;  
    case(state)
        INCREMENTA: begin 
            if(cnt == 99)
                state_nxt = RESETA;
            else begin
                cnt_nxt = cnt+'h1;
                state_nxt = INCREMENTA;
        end    
        RESETA: begin 
            cnt_nxt = 'h0;
            state_nxt = INCREMENTA;
        end
        DEFAULT: begin
            state_nxt = RESETA;
        end
    endcase
end

always @(posedge clk) begin
    if(reset) begin
        cnt <= 0;
        state <= INCREMENTA;
    end
    else begin
        cnt <= cnt_nxt;
        state <= state_nxt;
    end
end
\end{verilogcode}
%\end{listing}

Nesse exemplo, o circuito combinacional calcula os valores de
\ssf{state} e \ssf{cnt} que serão atualizados na próxima borda de
subida do sinal do relógio pelo circuito sequencial.

\item Não utilize atribuições com atrasos pois são normalmente
ignoradas pelas ferramentas de síntese, o que pode resultar num
comportamento inesperado do projeto.

\begin{verilogcode}
assign #10 Q = 0; // do not use
\end{verilogcode}

\item Utilize blocos \ssf{generate} para gerar várias instâncias de
módulos ou blocos de código. Na \secstr~\ref{sec:impl.mem}
utilizamos \ssf{generate} para criar o vetor com os bits de paridade
dos dados a serem gravados na memória;

\item Outras dicas podem ser encontradas nas páginas Web mencionadas
e livros especializados.

\end{itemize}


