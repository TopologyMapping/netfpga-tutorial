\section{Conclusões e perspectivas}

Neste trabalho apresentamos a plataforma NetFPGA através da
demonstração passo-a-passo do desenvolvimento de uma aplicação real.
A aplicação consiste de um \emph{firewall} para filtragem de pacotes
em \emph{hardware}. Os conceitos básicos da arquitetura da NetFPGA
foram discutidos. Mostramos os principais componentes de
\emph{hardware} e \emph{software}. Os aplicativos de referência como
placa de rede, comutador e roteador foram discutidos, além dos
módulos de interação \emph{hardware-software} como a interface de
registradores e o controlador de memória.

Dado o seu potencial no processamento de pacotes, consideramos que a
NetFPGA possui um grande potencial no desenvolvimento de novos
projetos de pesquisa, seja permitindo prover novas funcionalidades
no plano de dados, seja como ferramenta para o desenvolvimento de
novos padrões e protocolos de comunicação, seja no desenvolvimento
de novos protótipos de pesquisa.  Certamente, a NetFPGA ajudará no
desenvolvimento de trabalhos interessantes e com grande potencial na
área de redes de computadores.
