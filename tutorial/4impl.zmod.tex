\subsection{Modificação de pacotes em trânsito}
\label{sec:impl.mod}

Uma das funcionalidades da NetFPGA é a capacidade de modificar pacotes
em trânsito.  Iremos ilustrar essa funcionalidade decrementando o tempo
de vida (\emph{time to live}, TTL) do pacote.  Iremos também recalcular
o \emph{checksum} do pacote.  Para tanto vamos estender o código
apresentado na subseção anterior.

O tempo de vida do pacote é transmitido na terceira palavra
(figura~\ref{tab:impl.state.pktwords}).  Nosso \emph{firewall} processa
a terceira palavra do pacote no quarto estado (\ssf{WORD4\_ADDR\_TTL}).
Para decrementar o tempo de vida, iremos modificar o dado do pacote
antes de inseri-lo na fila de saída, como segue:

\begin{verilogcode}
      WORD3_CHECK_TCP_TTL: begin
         if (!in_fifo_empty && out_rdy) begin
            if(in_fifo_data[7:0] == 8'h06) begin
               in_fifo_rd_en = 1;
               out_fifo_wr = 1;
               out_fifo_din = {in_fifo_ctrl, in_fifo_data[63:16],
                               in_fifo_data[15:8] - 8'h1,
                               in_fifo_data[7:0]};
               state_next = WORD4_ADDR_CHKSUM;
            end
            else
               state_next = EMPTY_OUT_FIFO;
         end
         else
            state_next = WORD3_CHECK_TCP_TTL;
      end
\end{verilogcode}

De forma similar, precisamos atualizar o \emph{checksum} do pacote
devido ao decremento do tempo de vida.  Como o \emph{checksum} do
protocolo IP é simplesmente uma soma, podemos atualizá-lo simplesmente
somando o que foi subtraído devido ao decremento do tempo de
vida.\footnotemark{} No código abaixo, \ssf{chksum} possui 17~bits para
conseguirmos somar o \emph{carry out} em \ssf{\tt chksum\_cout}, que
possui 16~bits.

\footnotetext{O cálculo do \emph{checksum} do protocolo IP é detalhado
no RFC1071.  Ele depende do valor da soma das palavras de 2~bytes dos
campos do cabeçalho usando complemento de um.  Como o tempo de vida tem
apenas 1~byte e está alinhado com o byte mais significativo da palavra
de 2~bytes que o contém, nós incrementamos o byte mais significativo do
\emph{checksum}, somando \sssf{0x0100}, para compensar.}

\begin{verilogcode}
      assign chksum = {0, in_fifo_data[63:48]} + 16'h0100;
      assign chksum_cout = chksum[15:0] + {15'h0, chksum[16]};
        ...
      WORD4_ADDR_CHKSUM: begin
        if (!in_fifo_empty && out_rdy) begin
            in_fifo_rd_en = 1;
            out_fifo_wr = 1;
            in_out_fifo_dout = {in_fifo_ctrl, chksum_cout,
                        in_fifo_data[47:0]};
            state_next = WORD5_TCP_PORT;
        end
        else
            state_next = WORD4_ADDR_CHKSUM;
      end
\end{verilogcode}
