\section*{Apêndice A: Diretivas de programação}
\label{sec.impl.apendice}

Neste material adotamos algumas convenções para facilitar o
entendimento, listadas a seguir. Alguns dos exemplos citados são
recomendações de boas técnicas comuns em linguagens de programação,
enquanto outras são necessárias para que o hardware resultante alcance o
comportamento desejado do projeto.\footnote{As recomendações foram
extraídas e adaptadas de
\sssf{http://www.xilinx.com/itp/xilinx10/books/docs/sim/sim.pdf} e
\sssf{https://github.com/NetFPGA/netfpga/wiki/VerilogCodingGuidelines}.}

\begin{itemize}

\item Use constantes e parâmetros para conferir legibilidade ao código.
Defina constantes em letra \ssf{MAIÚSCULA}.

\item Trechos do código que fazem asserções, monitoram ou imprimem
valores de variáveis para teste do projeto devem estar posicionadas
entre as diretivas \ssf{synthesis translate\_off} e \ssf{synthesis
translate\_on}. Estes trechos são executados na simulação, mas são
ignorados pelo sintetizador na criação do \emph{bitfile} do projeto.

\item Em circuitos sequenciais utilize somente atribuições não-blocantes
(\ssf{<=}) dessa forma todas as atribuições dentro do bloco (geralmente
\ssf{always}) são avaliadas primeiro e depois realizadas na simulação e
permitem ao sintetizador inferir o circuito síncrono. No exemplo a
seguir todas as atribuições são sincronizadas com o sinal de relógio e
recebem os valores anteriores dos sinais, ou seja, a sequência não
importa:

\begin{verilogcode}
always @(posedge clk) begin
    B <= A;
    C <= B;
    D <= C;
end
\end{verilogcode}

\item Em circuitos combinacionais certifique-se de preencher a lista de
sensibilidade com todas as variáveis de interesse ou usar \ssf{always
@(*)}. Nesses circuitos qualquer alteração em alguma das variáveis na
lista forçam a reexecução do bloco (em geral \ssf{always} ou
\ssf{assign}) imediatamente. Certifique-se também de somente utilizar
atribuições blocantes (\ssf{=}) para que o sintetizador infira um
circuito assíncrono. O parâmetro \ssf{@(*)} é um recurso do Verilog que
preenche automaticamente a lista de sensibilidade com todos os valores
atribuídos dentro do bloco. No exemplo a seguir, as variáveis B, C e D
recebem o valor de A:

\begin{verilogcode}
always @(*) begin
    B = A;
    C = B;
    D = C;
end
\end{verilogcode}

\item Atribua todas as variáveis em todos os estados possíveis do seu
projeto. Este erro ocorre geralmente em expressões condicionais \ssf{if}
sem a cláusula \ssf{else} ou blocos \ssf{case}. No exemplo abaixo,
\ssf{cnt\_nxt} recebe o valor atual de \ssf{cnt} por padrão, para evitar
que ele deixe de ser atribuído se algum estado não requerer sua
atualização, e \ssf{state\_nxt} é atribuído em todos os estados. 

\item Em lógicas de controle é recomendado criar circuitos
combinatoriais que recebam os valores do próximo estado. Os valores do
próximo estado devem ser guardados em registradores em circuitos
síncronos com a borda do relógio:

\begin{verilogcode}
always @(*) begin
    cnt_nxt = cnt;  
    case(state)
        INCREMENTA: begin 
            if(cnt == 99)
                state_nxt = RESETA;
            else begin
                cnt_nxt = cnt+'h1;
                state_nxt = INCREMENTA;
        end    
        RESETA: begin 
            cnt_nxt = 'h0;
            state_nxt = INCREMENTA;
        end
        DEFAULT: begin
            state_nxt = RESETA;
        end
    endcase
end

always @(posedge clk) begin
    if(reset) begin
        cnt <= 0;
        state <= INCREMENTA;
    end
    else begin
        cnt <= cnt_nxt;
        state <= state_nxt;
    end
end
\end{verilogcode}
%\end{listing}

Nesse exemplo, o circuito combinacional calcula os valores de
\ssf{state} e \ssf{cnt} que serão atualizados na próxima borda de subida
do sinal do relógio pelo circuito sequencial.

\item Não utilize atribuições com atrasos pois são normalmente ignoradas
pelas ferramentas de síntese, o que pode resultar num comportamento
inesperado do projeto.

\begin{verilogcode}
assign #10 Q = 0; // do not use
\end{verilogcode}

\item Utilize blocos \ssf{generate} para gerar várias instâncias de
módulos ou blocos de código. Na \secstr~\ref{sec:impl.mem} utilizamos
\ssf{generate} para criar o vetor com os bits de paridade dos dados a
serem gravados na memória;

\item Outras dicas podem ser encontradas nas páginas Web mencionadas e
livros especializados.

\end{itemize}





\begin{comment}
\mr{movido da introdução}
\pg{O código abaixo escreve zeros em todos os endereços da memoria}
\begin{verbatim}
module reseta_sram #(
		parameter SRAM_DATA_WIDTH = 72,
		parameter SRAM_ADDR_WIDTH = 19
)
(
    output reg [SRAM_DATA_WIDTH-1:0] 	sram_data,
    output reg [SRAM_ADDR_WIDTH-1:0] 	sram_addr,
    output reg [7:0] 					sram_bw,
    output reg 							sram_we,
    output reg 							sram_tri_en,
	input  				    			clk,
	input								reset
    );

	reg									sram_tri_en_1, sram_tri_en_2;
	reg	[SRAM_DATA_WIDTH-1:0]	sram_data_1, sram_data_2;
	   
always @(posedge clk) begin
    if(reset) begin
        sram_addr <= 'h0;
        sram_data_2 <= 'h0;
        sram_bw <= 'h00; 
        sram_we <= 1'b0;
        sram_tri_en_2 = 1'b1;
    end
    else if(sram_addr < {SRAM_ADDR_WIDTH{1'b1}}) begin
        sram_addr <= sram_addr+'b1;
        sram_data_2 <= 'h0;
	    sram_bw <= 8'h00;		  
        sram_we <= 1'b0; 
        sram_tri_en_2 = 1'b1;
    end
    else begin
        sram_bw <= 'hff; 
        sram_we <= 1'b1; 
        sram_tri_en <= 1'b0; 
    end
    //estágio 1 do pipeline de escrita
    sram_tri_en_1 <= sram_tri_en_2;
    sram_tri_en <= sram_tri_en_1;
    //estágio 2 do pipeline de escrita
    sram_data_1 <= sram_data_2;
    sram_data <= sram_data_1;
end
endmodule

\end{verbatim}

Um arquivo de testes é necessário para visualizar o correto funcionamento:
\begin{verbatim}
module uut_reseta_sram #(

    parameter 	PERIOD = 10,
	parameter	SRAM_DATA_WIDTH = 72,
	parameter	SRAM_ADDR_WIDTH = 5
);
	// Inputs
	reg clk;
	reg reset;

	// Outputs
	wire [SRAM_DATA_WIDTH-1:0] 	sram_data;
	wire [SRAM_ADDR_WIDTH-1:0] 	sram_addr;
	wire [7:0] 							sram_bw;
	wire 									sram_we;
	wire 									sram_tri_en;
	
	reg [SRAM_DATA_WIDTH-1:0]		memoria [0:(1<<SRAM_ADDR_WIDTH)-1];
	integer 		idx;

	// Instantiate the Unit Under Test (UUT)
	reseta_sram  #(
	    .SRAM_DATA_WIDTH(SRAM_DATA_WIDTH),
	    .SRAM_ADDR_WIDTH(SRAM_ADDR_WIDTH)
    ) uut (
        .sram_data(sram_data),
        .sram_addr(sram_addr),
        .sram_bw(sram_bw),
        .sram_we(sram_we), 
        .sram_tri_en(sram_tri_en),
        .clk(clk),
        .reset(reset)
    );

	initial begin
        $dumpfile("sram.vcd");
        for(idx=0;idx<(1<<SRAM_ADDR_WIDTH);idx=idx+1)
            $dumpvars(0,memoria[idx]);
        $dumpvars(0,uut_reseta_sram);
        clk = 0;
        reset = 0;
        #5 reset = 1;
        #30 reset = 0;
        #1000 $finish;
    end

    always begin
        clk = 1'b0;
        #(PERIOD/2) clk = 1'b1;
        #(PERIOD/2);
    end  
	
	always @(*) begin
        if(!sram_we && sram_tri_en) begin
            if(sram_bw[0])
                memoria[sram_addr][8:0] <= memoria[sram_addr][8:0];
            else	
                memoria[sram_addr][8:0] <= sram_data[8:0];
            if(sram_bw[1])
                memoria[sram_addr][17:9] <= memoria[sram_addr][17:9];
            else
                memoria[sram_addr][17:9] <= sram_data[17:9];
            if(sram_bw[2])
                memoria[sram_addr][26:18] <= memoria[sram_addr][26:18];
            else
                memoria[sram_addr][26:18] <= sram_data[26:18];
            if(sram_bw[3])
                memoria[sram_addr][35:27] <= memoria[sram_addr][35:27];
            else
                memoria[sram_addr][35:27] <= sram_data[35:27];
            if(sram_bw[4])
                memoria[sram_addr][44:36] <= memoria[sram_addr][44:36];
            else
                memoria[sram_addr][44:36] <= sram_data[44:36];
            if(sram_bw[5])
                memoria[sram_addr][53:45] <= memoria[sram_addr][53:45];
            else
                memoria[sram_addr][53:45] <= sram_data[53:45];
            if(sram_bw[6])
                memoria[sram_addr][63:54] <= memoria[sram_addr][63:54];
            else
                memoria[sram_addr][63:54] <= sram_data[63:54];
            if(sram_bw[7])
                memoria[sram_addr][71:63] <= memoria[sram_addr][71:63];
            else
                memoria[sram_addr][71:63] <= sram_data[71:63];				
        end
    end
endmodule
\end{verbatim}


\pg{O arquivo de testes instancia o módulo para limpar a SRAM e demonstra o resultado da escrita, além de permitir a visualização em forma de onda das mudanças do conteúdo da memória. Essa simulação pode ser gerada, por exemplo, através das ferramentas Icarus Verilog e GTK-Wave que podem ser instaladas pelo yum através de:}

\begin{verbnobox}
sudo yum install iverilog
sudo yum install gtkwave
\end{verbnobox}

O comando a seguir invoca o compilador e gera um arquivo binário para ser executado pelo simulador:
\begin{verbnobox}
iverilog uut_reseta_sram reseta_sram -o srambin
\end{verbnobox}

O simulador gerará um arquivo de saída de nome sram.vcd, conforme definido no comando \$dumpfile:

\begin{verbnobox}
vvp srambin
\end{verbnobox}

Você verá uma mensagem indicando que o arquivo com as formas de onda foi criado:

\begin{verbnobox}
VCD info: dumpfile sram.vcd opened for output.
\end{verbnobox}

Agora invocamos o gtkwave para ler o arquivo gerado com o comportamento das variáveis:

\begin{verbnobox}
gtkwave sram.vcd
\end{verbnobox}

A simulação do projeto em formas de ondas é útil para identificarmos erros no comportamento em módulos isolados ou em conjuntos de módulos que possam ser encapsulados numa interface bem definida, entretanto esta tarefa não é trivial quando o projeto é constituído de dezenas de módulos. A NetFPGA possui enfoque em testes de regressão \pg{citar pagina que explica os testes de regressao}. Ela oferece funcionalidades que permitem testes extensivos sobre o projeto. %, o que normalmente não é trivial para ser feito através das formas de onda devido à dificuldade em rastrear as variáveis internas de todos os módulos instanciados no projeto, ou mesmo de se descrever o comportamento do projeto final a partir delas.
\end{comment}
